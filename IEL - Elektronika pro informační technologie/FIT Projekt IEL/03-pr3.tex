\section{Příklad 3}
% Jako parametr zadejte skupinu (A-H)
\tretiZadani{A}
Dle metody uzlových napětí si sestavím rovnice proudů (II. Kirchoffův zákon) a to podle uzlů A, B, C 
\begin{gather*}
A: I_{R1} + I_B - I_{R2} = 0\\
B: I_1 - I_{R3} - I_{R5} = 0\\
C: I_2 - I_1 - I_{R4} + I_{R5} = 0\\
\end{gather*}
\\
\begin{center}
Proudy pro jednotlivé rezistory s napětími $U_A$, $U_B$, $U_C$:
\end{center}
\begin{gather*}
I_{R1} =\frac{U - U_A}{R_1}\\
I_{R2} =\frac{U_A}{R_2}\\
I_{R3} =\frac{U_B - U_A}{R_3}\\
I_{R4} =\frac{U_C}{R_4}\\
I_{R5} =\frac{U_B - U_C}{R_5}\\
\end{gather*}
\\
\newpage
\begin{center}
Teď mohu dosadit do všech tří rovnic všechny známé hodnoty a vyjádřit si neznámá napětí:
\end{center}
\begin{gather*}
A: \frac{120 - U_A}{53} + \frac{U_B - U_A}{65} - \frac{U_A}{49} = 0\\
B: 0.9 - \frac{U_B - U_A}{65} - \frac{U_B - U_C}{32} = 0\\
C: 0.7 - 0.9 - \frac{U_C}{39} + \frac{U_B - U_C}{32} = 0\\
\end{gather*}
\\
\begin{center}
Nejprve si rovnici mírně upravím - vyjádřím si zlomky:
\end{center}
\begin{gather*}
\frac{-U_A}{53} + \frac{U_B}{65} - \frac{U_A}{65} - \frac{U_A}{49} = -\frac{120}{53}\\
\frac{U_A}{65} - \frac{U_B}{65} - \frac{U_B}{32} + \frac{U_C}{32} = -0.9\\
\frac{U_B}{32} - \frac{U_C}{32} - \frac{U_C}{39} = 0.2\\
\end{gather*}
\\
\begin{gather*}
-U_A \cdot(\frac{1}{53} + \frac{1}{65} + \frac{1}{49}) + U_B \cdot \frac{1}{65} = -\frac{120}{53}\\
U_A \cdot \frac{1}{65} - U_B \cdot (\frac{1}{32} + \frac{1}{65}) + U_C \cdot \frac{1}{32} = - 0.9\\
U_B \cdot \frac{1}{32} - U_C \cdot (\frac{1}{32} + \frac{1}{39}) = 0.2\\
\end{gather*}
\\
\begin{center}
Teď si všechno převedu do matice (sloupce = neznámé, poslední sloupec je číslo za rovnítkem),
řádky vynásobím tak, abych se zbavil zlomků a vypočítám UA, UB, UC převedením na jednotkovou
matici:
\end{center}
\begin{gather*}
\begin{pmatrix}
\begin{array}{ccc|c}
 -9 227 & 2597 & 0 &  -382 200 \\
 32 & -97 & 65  &  -1 872 \\
 0 & 39 & -71 & 249.6 \\
\end{array}
\end{pmatrix}
\end{gather*}
\\
\begin{center}
Po úpravách na jednotkovou matici dostanu tento tvar matice:
\end{center}
\begin{gather*}
\begin{pmatrix}
\begin{array}{ccc|c}
 1 & 0 & 0 &   57.4031 \\
 0 & 1 & 0 &   56.7803 \\
 0 & 0 & 1 &   27.6737 \\
\end{array}
\end{pmatrix}
\end{gather*}
\\
\newpage
\begin{center}
Výsledné hodnoty napětí jsou následovné:
\end{center}
\begin{gather*}
U_A = 57.4031 \text{V}\\
U_B = 56.7803 \text{V}\\
U_C = 27.6737 \text{V}\\
\end{gather*}
\\
\begin{center}
Vypočítám si $U_{R2}$, a potom dopočítám proud $I_{R2}$- vše dle vzorce, který jsem si vyjádřil hned nazačátku příkladu:
\end{center}
\begin{gather*}
I = \frac{U}{R} =>I_{R2} = \frac{U_{R2}}{R_2} = \frac{U_A}{R_2} => U_{R2} = U_A\\
\\
U_{R2} = 57.4031 \text{V}\\
\\
I_{R2} = \frac{57.4031}{49} = 1,1715 \text{A}\\
\end{gather*}